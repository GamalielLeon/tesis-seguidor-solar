\chapter{Resumen}


\textbf{Resumen:} En el presente reporte técnico se muestra el desarrollo de un sistema mecatrónico que tiene por objetivo principal apuntar hacia la posición relativa del Sol desde la superficie terrestre, con las características necesarias que le permitan tener uso industrial. Éste deberá de generar al menos 1 KWh de energía eléctrica mediante un sistema fotovoltaico.\\

Para la elaboración del presente trabajo se utilizó una metodología de diseño mecatrónico, la cual permite la concurrencia y sinergia de las disciplinas de la Mecatrónica, así como la integración y validación de cada uno de los subsistemas que conforman un sistema mecatrónico. Del mismo modo se hizo uso de diferentes herramientas de diseño para la selección de conceptos, dispositivos y métodos para detallar el diseño del sistema.\\

%El trabajo se desarrollará en diferentes etapas. La primera constará de una investigación de los conceptos necesarios para definir los alcances y funcionamiento del sistema a diseñar. En la segunda etapa se llevará a cabo el diseño, modelado, simulación, y validación de cada uno de los módulos que compondrán al sistema, así como la integración de éstos. La tercera etapa será la construcción física de cada uno de los módulos del sistema. Una vez hecho esto, se realizarán pruebas y mediciones, para luego hacer un análisis y comparación con los resultados obtenidos en el diseño, con el fin de verificarlos. La última etapa consistirá en integrar todos los módulos para conformar el sistema completo, para finalmente efectuar pruebas finales y así verificar la correcta integración.\\ %

\textbf{Palabras Clave:} Energía renovable, seguidor, sistema fotovoltaico. \\


%%%%%%fin del archivo

\endinput