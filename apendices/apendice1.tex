\chapter{Apéndice A. Descripción de criterios.}
% Table generated by Excel2LaTeX from sheet 'Descripcion criterios'
\begin{table}[H]
	\centering
	\footnotesize
	\caption{Descripción criterios módulos energético y mando central.}
	\begin{tabular}{|c|c|p{10em}|p{23em}|}
		\hline
		\multicolumn{1}{|p{1.715em}|}{\textbf{\footnotesize No.}} & \multicolumn{2}{p{9em}|}{\textbf{ Criterio}} & \textbf{ Descripción} \\
		\hline
		\hline
		\textbf{1} & \multicolumn{1}{c|}{\multirow{4}[26]{*}{\begin{sideways}\textbf{ Energético}\end{sideways}}} &  Facilidad de ensamble. &  Es la sencillez con la que se puede fijar los componentes. \\
		\cline{1-1}\cline{3-4}    \textbf{2} &       &  Accesibilidad a los componentes/piezas. &  Es la facilidad para cambiar y/o manipular los componentes. \\
		\cline{1-1}\cline{3-4}    \textbf{3} &       &  Accesibilidad al medio de almacenamiento. &  Es la facilidad para cambiar y/o manipular el medio de almacenamiento. \\
		\cline{1-1}\cline{3-4}    \textbf{4} &       &  Pérdidas de energía. &  Es la disminución de energía debido a la distancia entre los componetes y el medio de almacenamiento. \\
		\hline
		\textbf{ 5} & \multicolumn{1}{c|}{\multirow{9}[20]{*}{\begin{sideways}\textbf{ Mando central}\end{sideways}}} &  Accesibilidad a los componentes. &  Es la facilidad para cambiar y/o manipular los componentes. \\
		\cline{1-1}\cline{3-4}    \textbf{ 6} &       &  Capacidad de almacenamiento de información. &  Es la cantidad de información que se puede guardar, entre mayor, mejor. \\
		\cline{1-1}\cline{3-4}    \textbf{ 7} &       &  Capacidad de almacenamiento de información. &  Es la rapidez con la que puede ejecutar instruccciones la unidad de procesamiento central. \\
		\cline{1-1}\cline{3-4}    \textbf{ 8} &       &  Complejidad de la comunicación interna. &  Es la dificultad para interconectar y transmitir información entre los componentes internos de mando central. \\
		\cline{1-1}\cline{3-4}    \textbf{ 9} &       &  Facilidad de identificación de fallas. &  Rápida identificación del error y/o falla. \\
		\cline{1-1}\cline{3-4}    \textbf{ 10} &       &  Topología. &  Es el tipo de ramificaciones por las cuales se comunicarán los dispositivos. \\
		\cline{1-1}\cline{3-4}    \textbf{ 11} &       &  Esquema de comunicación. &  El tipo de comunicación que manejarán los dispositivos, ya sea punto a punto, maestro/esclavo, cliente/servidor. \\
		\hline	
	\end{tabular}%
	\label{tab:addlabel1}%
\end{table}%

\newpage

% Table generated by Excel2LaTeX from sheet 'Descripcion criterios (2)'
\begin{table}[H]
	\centering
	\small
	\caption{Descripción criterios módulos mando central, interfaz y movimiento azimutal.}
	\begin{tabular}{|c|c|p{10em}|p{21em}|}
		\hline
		\multicolumn{1}{|p{1.715em}|}{\textbf{\footnotesize No.}} & \multicolumn{2}{p{9em}|}{\textbf{ Criterio}} & \textbf{ Descripción} \\
		\hline
		\hline
		\textbf{12} & \multicolumn{1}{c|}{\multirow{3}[15]{*}{\begin{sideways}\textbf{ Mando central}\end{sideways}}} &  Velocidad de transmisión de información. &  Es la rapidez con la que el protocolo puede enviar y recibir la información de un lugar a otro. \\
		\cline{1-1}\cline{3-4}    \textbf{ 13} &       &  Protección Industrial. &  Grado de protección contra polvo y agua. \\
		\cline{1-1}\cline{3-4}    \textbf{ 14} &       &  Complejidad de implementación. &  Es la dificultad al implementar varios modos de operación. \\
		\hline
		\textbf{ 15} & \multicolumn{1}{c|}{\multirow{3}[32]{*}{\begin{sideways}\textbf{ Interfaz}\end{sideways}}} &  Manipulabilidad del sistema por parte del usuario. &  Es el grado de interacción y control que el usuario puede ejercer sobre el sistema. \\
		\cline{1-1}\cline{3-4}    \textbf{ 16} &       &  Resistencia a la corrosión y temperatura. &  Es la capacidad de funcionar a ciertas condiciones ambientales y/o del entorno sin perder eficiencia ni reducir su rendimiento. \\
		\cline{1-1}\cline{3-4}    \textbf{ 17} &       &  Forma del protector. &  Es la estructura de los elementos que contendrán a la interfaz. \\
		\hline
		\textbf{ 18} & \multicolumn{1}{c|}{\multirow{9}[40]{*}{\begin{sideways}\textbf{ Movimiento azimutal}\end{sideways}}} &  Accesibilidad al sensor. &  Es la facilidad para cambiar y/o manipular el sensor. \\
		\cline{1-1}\cline{3-4}    \textbf{ 19} &       &  Facilidad de fijación. &  Es la sencillez para colocar y asegurar el sensor. \\
		\cline{1-1}\cline{3-4}    \textbf{ 20} &       &  Precisión. &  Alta resolución de medición. \\
		\cline{1-1}\cline{3-4}    \textbf{ 21} &       &  Par transmitido. &  Es el brazo de palanca transmitido para realizar el movimiento empleando el menor esfuerzo. \\
		\cline{1-1}\cline{3-4}    \textbf{ 22} &       &  Relación de reducción. &  Es la reducción de la distancia recorrida en la entrada reflejada en la salida. \\
		\cline{1-1}\cline{3-4}    \textbf{ 23} &       &  Dimensiones. &  Es el espacio que ocupan los elementos del mecanismo de reducción. \\
		\cline{1-1}\cline{3-4}    \textbf{ 24} &       &  Dimensiones. &  Es el espacio que ocupan los elementos del mecanismo de transmisión. \\
		\cline{1-1}\cline{3-4}    \textbf{ 25} &       &  Deslizamiento. &  Es la facilidad de que resbale el medio de transmisión, entre menos, mejor. \\
		\cline{1-1}\cline{3-4}    \textbf{ 26} &       &  Par transmitido. &  Es el brazo de palanca transmitida para realizar el movimiento empleando el menor esfuerzo. \\
		\hline
	\end{tabular}%
	\label{tab:addlabel2}%
\end{table}%

\newpage

% Table generated by Excel2LaTeX from sheet 'Descripcion criterios (2)'
\begin{table}[H]
	\centering
	\small
	\caption{Descripción criterios módulos movimiento de elevación y estructural.}
	\begin{tabular}{|c|c|p{10em}|p{21em}|}
		\hline
		\multicolumn{1}{|p{1.715em}|}{\textbf{\footnotesize No.}} & \multicolumn{2}{p{9em}|}{\textbf{Criterio}} & \textbf{Descripción} \\
		\hline
		\hline
		\textbf{ 27} & \multicolumn{1}{c|}{\multirow{10}[30]{*}{\begin{sideways}\textbf{ Movimiento de elevación}\end{sideways}}} & \multicolumn{1}{l|}{ Accesibilidad al sensor.} & \multicolumn{1}{l|}{ Es la facilidad para cambiar y/o manipular el sensor.} \\
		\cline{1-1}\cline{3-4}    \textbf{ 28} &       &  Facilidad de fijación. &  Es la sencillez para colocar y asegurar el sensor. \\
		\cline{1-1}\cline{3-4}    \textbf{ 29} &       &  Precisión. &  Alta resolución de medición. \\
		\cline{1-1}\cline{3-4}    \textbf{ 30} &       &  Par transmitido. &  Es el brazo de palanca transmitida para realizar el movimiento empleando el menor esfuerzo. \\
		\cline{1-1}\cline{3-4}    \textbf{ 31} &       &  Relación de Reducción. &  Es la reducción de la distancia recorrida en la entrada reflejada en la salida. \\
		\cline{1-1}\cline{3-4}    \textbf{\ 32} &       &  Dimensiones. &  Es el espacio que ocupan los elementos del mecanismo de reducción. \\
		\cline{1-1}\cline{3-4}    \textbf{ 33} &       &  Autobloqueo. &  La capacidad que tiene un mecanismo para mantenerse en una posición cuando no se transmite movimiento. \\
		\cline{1-1}\cline{3-4}    \textbf{ 34} &       &  Dimensiones. &  Es el espacio que ocupan los elementos del mecanismo de transmisión. \\
		\cline{1-1}\cline{3-4}    \textbf{ 35} &       &  Deslizamiento. &  Es la facilidad de que resbale el medio de transmisión, entre menos, mejor. \\
		\cline{1-1}\cline{3-4}    \textbf{\ 36} &       &  Par transmitido. &  Es el brazo de palanca transmitida para realizar el movimiento empleando el menor esfuerzo. \\
		\hline
		\textbf{37} & \multicolumn{1}{c|}{\multirow{3}[33]{*}{\begin{sideways}\textbf{Estructural}\end{sideways}}} & Dimensiones. & Es el espacio máximo que puede ocupar el seguidor solar. \\
		\cline{1-1}\cline{3-4}    \textbf{38} &       & Deterioro por exposición a las condiciones ambientales (lluvia, polvo, humedad, calor, viento, etc.). & Es la degradación de los componentes debido a su exposición al ambiente, entre menos sea, mejor. \\
		\cline{1-1}\cline{3-4}    \textbf{39} &       & Normalizado. & Cumplimiento con normas de estandarización comerciales. \\
		\cline{1-1}\cline{3-4}    \textbf{40} &       & Superficie de la base. & Es el área que soportará el cuerpo del seguidor. \\
		\hline
	\end{tabular}%
	\label{tab:addlabel3}%
\end{table}%

\newpage
% Table generated by Excel2LaTeX from sheet 'Descripcion criterios (2)'
\begin{table}[H]
	\centering
	\small
	\caption{Descripción criterios módulo estructural y generales.}
	\begin{tabular}{|c|c|p{10em}|p{21em}|}
		\hline
		\multicolumn{1}{|p{1.715em}|}{\textbf{No.}} & \multicolumn{2}{p{8em}|}{\textbf{Criterio}} & \textbf{Descripción} \\
		\hline
		\hline
		\textbf{41} & \multicolumn{1}{c|}{\multirow{9}[50]{*}{\begin{sideways}\textbf{Estructural}\end{sideways}}} & Tipo de piso. & Es el tipo de suelo en el que será colocado el seguidor solar. \\
		\cline{1-1}\cline{3-4}    \textbf{42} &       & Estabilidad. & Es la firmeza que tendrá la disposición de los paneles ante diversos esfuerzos y vibraciones. \\
		\cline{1-1}\cline{3-4}    \textbf{43} &       & Facilidad de interconexión. & Es la sencillez para conectar los paneles. \\
		\cline{1-1}\cline{3-4}    \textbf{44} &       & Par transmitido. &  Brazo de palanca generado por la inclinación del colector. \\
		\cline{1-1}\cline{3-4}    \textbf{45} &       & Geometría de ensamble. & Es la facilidad para acoplar o juntar los paneles al alma del colector. \\
		\cline{1-1}\cline{3-4}    \textbf{46} &       & Libertad de giro completo. & El soporte permite que el sistema realice una revolución completa. \\
		\cline{1-1}\cline{3-4}    \textbf{47} &       & Accesibilidad de mantenimiento. & Mantenimiento menos laborioso. \\
		\cline{1-1}\cline{3-4}    \textbf{48} &       & Accesibilidad de componentes. & Es la facilidad para cambiar y/o manipular los componentes. \\
		\hline
		\textbf{49} & \multicolumn{1}{c|}{\multirow{6}[34]{*}{\begin{sideways}\textbf{Generales}\end{sideways}}} & Consumo energético. & Es la minimización del consumo interno de energía del sistema. \\
		\cline{1-1}\cline{3-4}    \textbf{50} &       & Modularidad. & Es la disminución de la dependencia entre componentes del sistema. \\
		\cline{1-1}\cline{3-4}    \textbf{51} &       & Error. & Es la disminución entre el valor presente y el valor esperado. \\
		\cline{1-1}\cline{3-4}    \textbf{52} &       & Costo(menor costo monetario). & Que se reduzca la inversión monetaria para la construcción del sistema. \\
		\cline{1-1}\cline{3-4}    \textbf{53} &       & Peso (más ligero). & Es la minimización de las cargas sobre la estructura del sistema. \\
		\cline{1-1}\cline{3-4}    \textbf{54} &       & Disponibilidad. & Suficiente cantidad de dispositivos en el mercado. \\
		\hline
	\end{tabular}%
	\label{tab:addlabel4}%
\end{table}%

\bigskip
%%%%%%fin del archivo
\endinput 