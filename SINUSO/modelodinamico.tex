
\section{Modelo Dinamico}

Miura y Shimoyama[23] estudiaron la aproximación del modelo dinámico de un bípedo a la de un péndulo invertido después Kajita [24] estudió la validación de este método a diferentes tipos de robots puesto que [23] trabajó con un caso en particular.\\

Se aproximó la dinamica del robot con el método de péndulo invertido. La masa está concentrada en el centro de masa (CoM) de el robot, y la base del péndulo coincide con el soporte del pie del robot, como se ilustra en la figura siguiente.


\begin{figure}[h]
	\centering
		\includegraphics{figs/dinamica.jpg}
	\caption{Pendulo Invertido}
	\label{fig:dinamica}
\end{figure}


Mediante las ecuaciones Lagrange:

\begin{equation}
	\tau_{x_a}(\theta)=m\left[g+\left(\frac{\tau_{x_a}(\theta)}{Lm}-gsen(\theta)\right)sen(\theta)-\frac{v^2}{L}cos(\theta)\right]x_a
\end{equation}
  
  
Cabe destacar que este modelo interactua solo con un torque máximo requerido dado por


  
  
\begin{equation}
	\tau_{x_a}\left(\theta\right)=\frac{mgx_{a}\left(1-sen^2\left(\theta\right)\right)}{1-\frac{sen\left(\theta\right)}{L}x_a}
\end{equation}

\begin{equation}
		\tau_{x_b}\left(\theta\right)=\frac{mgx_{b}\left(1-sen^2\left(\theta\right)\right)}{1-\frac{sen\left(\theta\right)}{L}x_b}
\end{equation}


Pero este modelo no considera las fuerzas laterales que son el resultado de conservar el equilibrio.

