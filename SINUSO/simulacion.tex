\chapter{Simulaci\'on} FTKPZ 