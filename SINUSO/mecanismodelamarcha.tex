
\subsection{Mecanismo de la Marcha.}

Si se plantea el mecanismo por el que se produce la marcha, se debe tener en cuenta que el cuerpo humano al caminar se comporta como un sistema físico y como un organismo biológico, y por consiguiente está sujeto a las leyes físicas del movimiento y a las leyes biológicas de la acción muscular. Se han estudiado los requerimientos energéticos de una persona adulta normal caminando a diferentes velocidades y se ha comprobado que toda persona tiene una velocidad de marcha de $4.5km/h$ la cual requiere un mínimo de energía, para ésto, es interesante analizar los desplazamientos que sufre el centro de gravedad del cuerpo que aproximadamente se sitúa por delante de la segunda vértebra sacra.\\

Durante la marcha, el cuerpo sufre un rítmico desplazamiento arriba-abajo, este desplazamiento vertical está en íntima relación con la locomoción bipodal: en las fases de doble apoyo el centro de gravedad está en el punto más bajo; en las fases de apoyo unilateral, el centro de gravedad alcanza su punto más alto, la distancia entre éstos dos puntos extremos es de 4 o 5 cm. También se ha comprobado que el centro de gravedad, en su desplazamiento, describe una curva sinusoidal, que es la que demanda menor consumo energético. Para conseguir éste desplazamiento existe una serie de movimientos coordinados de la extremidad inferior. La pelvis, la cadera y la rodilla actúan coordinadamente para disminuir la amplitud de la curva, mientras que la rodilla, el tobillo y el pie trabajan para suavizar el cambio de sentido de la curva.

\begin{figure}[h]
	\centering
		\fbox{\includegraphics{figs/dos.jpg}}
	\caption{Movimiento del centro de gravedad.}
	\label{fig:dos}
\end{figure}
   
   
  La pelvis contribuye al desplazamiento del centro de gravedad con dos movientos: el primero en el plano horizontal, que son los movimientos vistos desde arriba o desde abajo del cuerpo humano [14] y el otro en el plano vertical, que son los movimientos vistos desde adelante o atrás del cuerpo humano [14]. En el plano horizontal se realiza un movimiento de rotación parecido al movimiento de un compás, que puede desplazarse sin cambiar la altura de la cruz. Si no se tuviera éste movimiento, la caminata sería a modo del movimiento de unas tijeras, abriendo y cerrando las hojas. Al caminar se adoptan los dos tipos de movimiento, el del compás con la rotacion pélvica y el de las tijeras con la flexoextención de la cadera. Como lo muestran las figuras 1.13 y 1.14 respectivamente.\\
  
  
\begin{figure}
	\centering
		\fbox{\includegraphics[scale=1.3]{figs/tres.jpg}}
	\caption{Movimiento de la pelvis en el plano horizontal.}
	\label{fig:tres}
\end{figure}

  
  El movimiento en el otro plano consiste en la inclinación de la pelvis hacia el lado de la pierna oscilante al igual que el movimineto de rotación contribuye a disminuir el desplazamiento vertical del centro de gravedad. La rodilla también participa en la disminución del desplazamiento del centro de gravedad al estar en discreta flexión en el momento en que el cuerpo pasa por encima de la pierna que apoya.\\
  
  
\begin{figure}[h]
	\centering
	
	\fbox{	\includegraphics[scale=1.2]{figs/cuatro.jpg}}
	\caption{Movimiento frontal de la pelvis.}
	\label{fig:cuatro}
\end{figure}

  
  Si estos tres movimientos no sucedieran, el desplazamiento vertical del centro de gravedad sería dos veces mayor; ahora bien, si sólo participaran estos tres movimientos, la trayectoria descrita por el centro de gravedad sería de arcos interrumpidos. Si la caminata fuera con una rodilla rígida, sin tobillo ni pie, el choque en el paso produciría una desaceleración brusca del centro de gravedad, en pocas palabras, el movimiento  de la rodilla, pie y tobillo sirven en el descenso como un sistema de amortiguamiento que coincide como lo plantea [4], y en el despegue actúa como un sistema de aceleración del centro de gravedad.             
  